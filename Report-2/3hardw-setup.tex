\section{Detailed hardware setup}

The wireless network system consists of a Hub and sensor nodes. After careful examination of all available nodes the Multi-standard SensorTag from Texas instruments \cite{TIsensortag} was chosen.


\subsection{Sensor Node - TI Sensor Tag}

The General requirements for the node were design for durability, optimal functionality and user friendly application. This sensor application requires many precise design parameters from the node but those can not all be satisfied. Thus, the general requirements will be discussed and then shown how well the TI sensor tag fulfills them and how it could be added. With the time scope of the project, some of these requirements will not be met, rather a focus will be but on functionality of current components.

\subsubsection{General node requirements}

The nodes have to be small in size to fit even in small plant pots. They should be battery powered and in order to be able to acquire the necessary data, they need to have sensors in and above the soil. Moreover, the nodes should be waterproof, contain a status-LED and a RF-module for wireless communication with the Hub.

Each sensor node should be able to measure the following:
\begin{itemize}
	\item Brightness 
	\item Temperature
	\item Soil humidity
	\item Air humidity
	\item Ph-Value
\end{itemize}



However, a high sampling rate is not necessary and therefore keeping the sampling rate to a minimum increases battery life substantially. For the same reason it is advisable to decrease the transmission rate by only sending data upon significant change.

\subsubsection{}

\subsection{Hub - Raspberry Pi 3}
The Hub has to be able to collect all data provided from the nodes. It can be connected to the buildings power grid and Internet. A user interface (UI) should be digitally accessible via touchscreen, web page and/or app. Additionally the system should be able to inform the user about critical sensor values. To decide whether a value is critical or not, the Hub has to either know which type of plant the sensor monitors and then check a Internet database for specific requirements or rely on user defined values.
