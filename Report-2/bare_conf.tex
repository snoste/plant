
%% bare_conf.tex
%% V1.4b
%% 2015/08/26
%% by Michael Shell
%% See:
%% http://www.michaelshell.org/
%% for current contact information.
%%
%% This is a skeleton file demonstrating the use of IEEEtran.cls
%% (requires IEEEtran.cls version 1.8b or later) with an IEEE
%% conference paper.
%%
%% Support sites:
%% http://www.michaelshell.org/tex/ieeetran/
%% http://www.ctan.org/pkg/ieeetran
%% and
%% http://www.ieee.org/

%%*************************************************************************
%% Legal Notice:
%% This code is offered as-is without any warranty either expressed or
%% implied; without even the implied warranty of MERCHANTABILITY or
%% FITNESS FOR A PARTICULAR PURPOSE! 
%% User assumes all risk.
%% In no event shall the IEEE or any contributor to this code be liable for
%% any damages or losses, including, but not limited to, incidental,
%% consequential, or any other damages, resulting from the use or misuse
%% of any information contained here.
%%
%% All comments are the opinions of their respective authors and are not
%% necessarily endorsed by the IEEE.
%%
%% This work is distributed under the LaTeX Project Public License (LPPL)
%% ( http://www.latex-project.org/ ) version 1.3, and may be freely used,
%% distributed and modified. A copy of the LPPL, version 1.3, is included
%% in the base LaTeX documentation of all distributions of LaTeX released
%% 2003/12/01 or later.
%% Retain all contribution notices and credits.
%% ** Modified files should be clearly indicated as such, including  **
%% ** renaming them and changing author support contact information. **
%%*************************************************************************


\documentclass[conference]{IEEEtran}
\usepackage{cite}
\usepackage[pdftex]{graphicx}
 % \graphicspath{{../pdf/}{../jpeg/}}
 % \DeclareGraphicsExtensions{.pdf,.jpeg,.png}
\usepackage{amsmath}
%\usepackage{algorithmic}
%\usepackage{array}
%\usepackage{url}
% correct bad hyphenation here
\hyphenation{op-tical net-works semi-conduc-tor}


\begin{document}
\title{Report 2\\Related Work and Design}

\author{\IEEEauthorblockN{Filippo Bernardi}
\IEEEauthorblockA{Bologna University, Italy\\
Tongji University, China\\
TU/e, Netherlands\\
Email: filippobernardi@outlook.it}
\and
\IEEEauthorblockN{Snorri Stefánsson}
\IEEEauthorblockA{University of Reykjavik, Iceland\\
TU/e, Netherlands\\
Email: snorriste@gmail.com}
\and
\IEEEauthorblockN{Jonas Wallmeier}
\IEEEauthorblockA{Technical University Ilmenau, Germany\\
TU/e, Netherlands\\
Email: j.a.wallmeier@student.tue.nl}}

% make the title area
\maketitle

\begin{abstract}
Description of a wireless application for monitoring vital signs of plants and for supporting the user to maintain them alive. \\

\end{abstract}

% For peerreview papers, this IEEEtran command inserts a page break and
% creates the second title. It will be ignored for other modes.
\IEEEpeerreviewmaketitle
\hfill \today
\\



%(Application description and exact scenario including research and/or industrial motivation)
\section{Application description}
Keeping plants alive can be a very time consuming task: it requires knowledge about different types of plants as well as periodical checks if these requirements are fulfilled. 
The main idea behind this application is to create a network of nodes, placed directly in the soil of each plant, that, connected to a main hub, helps the user to keep all the plants healthy. Moreover, each nodes has a series of sensor to measure some of the environmental indexes, e.g. temperature or humidity. The main Hub, a device, receives all the sensors data from each node and stores it in a database. This hub uses the contributed data from each plants and checks if it's acceptable for the type of plant the user defined. The costumer can easily view all the sensors information which is stored in the Hub's memory and displayed in a user friendly platform on the web.	\\
This application is suitable for many type of buildings but in this paper it will be assumed to be operating in a large family house with a reasonable garden. The plant monitor system should work indoor and outdoor. Although not all plants get exposed to rain, every node should be waterproof were they are placed directly on the soil of the plant. However, in this project the robust final design will not be implemented in relations to water resistance and more focus is on protocols and application layer.\\
Each node has numerous sensor which will inform the user of necessary actions needed to be taken. In this paper we describe only the sensors available on sensor tag by Texas Instrument. In further development additional sensors could be added to the node.
 \\
 %Literature survey on similar applications (research or industry) and their solutions
 
\section{similar applications}
There are many plant monitoring system already available on the market. However, most of them use a smart phone as a host without creating a sensor network. It is important to bare in mind that the companies do not declare which protocols are used and therefore information is limited. Nevertheless, some assumptions could be reliably made from the data provided. The following devices demonstrate an overview of existing solutions:\\
\begin{enumerate}
	\item Edyn \\
This application is called Edyn, is composed by a device that has to be placed in the soil. Moreover, is has sensors for light, moisture, humidity and is able to inform its users about the nutrition status of the plant. There is a solar panel integrated on top of the device to power the device and charge its battery. The apparatus connects to a WiFi network and communicates with a smart phone via app where all the information is shown. The system uses a actuator, a automatic watering system, enabled upon information from its partner device.\\
\item PlantLink\\
PlantLink is a system which can sense the moisture of a plant and wirelessly communicate with a base station. This base station can be paired with up to 64 sensors nodes. There is a smartphone app that informs the users about the health levels of the plant. Each sensor is associated to a specific type of plant by the user and the system can then look up generic requirements for that plant and advice the user accordingly. The system also has a PlantLink valve, a specific actuator able to water the system automatically. The user can also control the actuator through the app if required. This application seems to be different from the amount of sensors in the system. Therefore, PlantLink is restricted to the humidity of the plant whereas the other solutions mentioned here provide multiple sensor values and meet more requirements of the users.\\

\item Parrot Flower Power\\
This third application available on the market is composed of a device that has to be placed directly on the soil of plant to monitoring, like the others. It does not have any actuator and reports to the user via smart phone app. The system is battery powered that can be easily changed and it also come with integrated light, moisture and temperature sensors. This device communicates straight to a smart phone via bluetooth.
\end{enumerate}

\subsection{Conclusion of existing devices}

The described solutions listed show the general purpose and requirements of a plant monitoring systems. User requirements are the most important aspect of each of these systems and have to be taken into account in the design. By looking at the those three systems, they all report the valuable information to the user in a comfortable way but some of them lack the off site connectivity and scalability of the amount of sensor nodes and range, which is vital for some plants in order for them to stay alive. 


\section{Design}
\subsection{Sensor Node}
The nodes have to be small in size to fit even in small plant pots. They should be battery powered and in order to be able to acquire the necessary data, they need to have sensors in and above the soil. Moreover, the nodes should be waterproof, contain a status-LED and a RF-module for wireless communication with the Hub.

Each sensor node should be able to measure the following:
\begin{itemize}
	\item Brightness
	\item Temperature
	\item Soil humidity
	\item Air humidity
	\item Ph-Value
\end{itemize}

However, a high sampling rate is not necessary and therefore keeping the sampling rate to a minimum increases battery life substantially. For the same reason it is advisable to decrease the transmission rate by only sending data upon significant change.

\subsection{Hub}
The Hub has to be able to collect all data provided from the nodes. It can be connected to the buildings power grid and Internet. A user interface (UI) should be digitally accessible via touchscreen, web page and/or app. Additionally the system should be able to inform the user about critical sensor values. To decide whether a value is critical or not, the Hub has to either know which type of plant the sensor monitors and then check a Internet database for specific requirements or rely on user defined values.


%\begin{figure}[!t]
%\centering
%\includegraphics[width=2.5in]{myfigure}
% where an .eps filename suffix will be assumed under latex, 
% and a .pdf suffix will be assumed for pdflatex; or what has been declared
% via \DeclareGraphicsExtensions.
%\caption{Simulation results for the network.}
%\label{fig_sim}
%\end{figure}

%\begin{figure*}[!t]
%\centering
%\subfloat[Case I]{\includegraphics[width=2.5in]{box}%
%\label{fig_first_case}}
%\hfil
%\subfloat[Case II]{\includegraphics[width=2.5in]{box}%
%\label{fig_second_case}}
%\caption{Simulation results for the network.}
%\label{fig_sim}
%\end{figure*}

%\begin{table}[!t]
%% increase table row spacing, adjust to taste
%\renewcommand{\arraystretch}{1.3}
% if using array.sty, it might be a good idea to tweak the value of
% \extrarowheight as needed to properly center the text within the cells
%\caption{An Example of a Table}
%\label{table_example}
%\centering
%% Some packages, such as MDW tools, offer better commands for making tables
%% than the plain LaTeX2e tabular which is used here.
%\begin{tabular}{|c||c|}
%\hline
%One & Two\\
%\hline
%Three & Four\\
%\hline
%\end{tabular}
%\end{table}
\begin{flushleft}
	
\section{Conclusion}
	The authors hope that this system will help people maintain their apartment green.
	The device described in this paper will be develop in  next two month creating a real prototype.
\end{flushleft}

\section{Bibliography}
\begin{itemize}
	\item https://www.edyn.com/
	\item https://myplantlink.com/how-it-works
	\item http://global.parrot.com/au/products/flower-power/
	
\end{itemize}



% that's all folks
\end{document}


