
%% bare_conf.tex
%% V1.4b
%% 2015/08/26
%% by Michael Shell
%% See:
%% http://www.michaelshell.org/
%% for current contact information.
%%
%% This is a skeleton file demonstrating the use of IEEEtran.cls
%% (requires IEEEtran.cls version 1.8b or later) with an IEEE
%% conference paper.
%%
%% Support sites:
%% http://www.michaelshell.org/tex/ieeetran/
%% http://www.ctan.org/pkg/ieeetran
%% and
%% http://www.ieee.org/

%%*************************************************************************
%% Legal Notice:
%% This code is offered as-is without any warranty either expressed or
%% implied; without even the implied warranty of MERCHANTABILITY or
%% FITNESS FOR A PARTICULAR PURPOSE! 
%% User assumes all risk.
%% In no event shall the IEEE or any contributor to this code be liable for
%% any damages or losses, including, but not limited to, incidental,
%% consequential, or any other damages, resulting from the use or misuse
%% of any information contained here.
%%
%% All comments are the opinions of their respective authors and are not
%% necessarily endorsed by the IEEE.
%%
%% This work is distributed under the LaTeX Project Public License (LPPL)
%% ( http://www.latex-project.org/ ) version 1.3, and may be freely used,
%% distributed and modified. A copy of the LPPL, version 1.3, is included
%% in the base LaTeX documentation of all distributions of LaTeX released
%% 2003/12/01 or later.
%% Retain all contribution notices and credits.
%% ** Modified files should be clearly indicated as such, including  **
%% ** renaming them and changing author support contact information. **
%%*************************************************************************


\documentclass[conference]{IEEEtran}

\usepackage[pdftex]{graphicx}
 % \graphicspath{{../pdf/}{../jpeg/}}
 % \DeclareGraphicsExtensions{.pdf,.jpeg,.png}
\usepackage{amsmath}
\usepackage{booktabs}
\usepackage{pgfgantt}
%\usepackage{algorithmic}
%\usepackage{array}
%\usepackage{url}
% correct bad hyphenation here
\hyphenation{op-tical net-works semi-conduc-tor}

\usepackage{biblatex}
\usepackage[colorlinks=true,allcolors=black]{hyperref}
%\usepackage[backend=biber, bibencoding=utf8, style=ieee]{biblatex}
\addbibresource{references.bib}


\begin{document}
\title{Report 2\\Related Work and Design}

\author{\IEEEauthorblockN{Filippo Bernardi}
\IEEEauthorblockA{Bologna University, Italy\\
Tongji University, China\\
TU/e, Netherlands\\
Email: filippobernardi@outlook.it}
\and
\IEEEauthorblockN{Snorri Stefansson}
\IEEEauthorblockA{University of Reykjavik, Iceland\\
TU/e, Netherlands\\
Email: snorriste@gmail.com}
\and
\IEEEauthorblockN{Jonas Wallmeier}
\IEEEauthorblockA{Technical University Ilmenau, Germany\\
TU/e, Netherlands\\
Email: j.a.wallmeier@student.tue.nl}}

% make the title area
\maketitle

\begin{abstract}
Description of a wireless application for monitoring vital signs of plants and for supporting the user to maintain them alive. \\

\end{abstract}

% For peerreview papers, this IEEEtran command inserts a page break and
% creates the second title. It will be ignored for other modes.
\IEEEpeerreviewmaketitle
\hfill \today
\\

%Application description and exact scenario including research and/or industrial motivation

%Literature survey on similar applications (research or industry) and their solutions


%Intended protocol stack and justification for tha

%Network architecture options

%The method and tools for implementation (simulation, hardware platform, ...)

%Hardware/software implementation plan

%I but this here to prevent the pagebreaks inbetween the include{file} :) happy happy hippo filippo
\begingroup
\let\clearpage\relax
%(Application description and exact scenario including research and/or industrial motivation)
\section{Application description}
Keeping plants alive can be a very time consuming task: it requires knowledge about different types of plants as well as periodical checks if these requirements are fulfilled. 
The main idea behind this application is to create a network of nodes, placed directly in the soil of each plant, that, connected to a main hub, helps the user to keep all the plants healthy. Moreover, each nodes has a series of sensor to measure some of the environmental indexes, e.g. temperature or humidity. The main Hub, a device, receives all the sensors data from each node and stores it in a database. This hub uses the contributed data from each plants and checks if it's acceptable for the type of plant the user defined. The costumer can easily view all the sensors information which is stored in the Hub's memory and displayed in a user friendly platform on the web.	\\
This application is suitable for many type of buildings but in this paper it will be assumed to be operating in a large family house with a reasonable garden. The plant monitor system should work indoor and outdoor. Although not all plants get exposed to rain, every node should be waterproof were they are placed directly on the soil of the plant. However, in this project the robust final design will not be implemented in relations to water resistance and more focus is on protocols and application layer.\\
Each node has numerous sensor which will inform the user of necessary actions needed to be taken. In this paper we describe only the sensors available on sensor tag by Texas Instrument. In further development additional sensors could be added to the node.
 \\

%Literature survey on similar applications (research or industry) and their solutions
\section{similar applications}
There are many plant monitoring system already available on the market. However, most of them use a smart phone as a host without creating a sensor network. It is important to bare in mind that the companies do not declare which protocols are used and therefore information is limited. Nevertheless, some assumptions could be reliably made from the data provided. The following devices demonstrate an overview of existing solutions:\\
\begin{enumerate}
	\item Edyn  \\
This application is called Edyn, is composed by a device that has to be placed in the soil. Moreover, is has sensors for light, moisture, humidity and is able to inform its users about the nutrition status of the plant. There is a solar panel integrated on top of the device to power the device and charge its battery. The apparatus connects to a WiFi network and communicates with a smart phone via app where all the information is shown. The system uses a actuator, a automatic watering system, enabled upon information from its partner device.\cite{Edyn}\\
\item PlantLink \\
PlantLink is a system which can sense the moisture of a plant and wirelessly communicate with a base station. This base station can be paired with up to 64 sensors nodes. There is a smartphone app that informs the users about the health levels of the plant. Each sensor is associated to a specific type of plant by the user and the system can then look up generic requirements for that plant and advice the user accordingly. The system also has a PlantLink valve, a specific actuator able to water the system automatically. The user can also control the actuator through the app if required. This application seems to be different from the amount of sensors in the system.\cite{MyPlantlink} Therefore, PlantLink is restricted to the humidity of the plant whereas the other solutions mentioned here provide multiple sensor values and meet more requirements of the users.\\

\item Parrot Flower Power\\
This third application available on the market is composed of a device that has to be placed directly on the soil of plant to monitoring, like the others. It does not have any actuator and reports to the user via smart phone app. The system is battery powered that can be easily changed and it also come with integrated light, moisture and temperature sensors. This device communicates straight to a smart phone via bluetooth. \cite{parrot}
\end{enumerate}

\subsection{Conclusion of existing devices}

The solutions listed previously show the general purpose and requirements of a plant monitoring systems. User requirements are the most important aspect of each of these systems and have to be taken into account in the design. By looking at the those three systems, they all report the valuable information to the user in a comfortable way but some of them lack the off site connectivity and scalability of the amount of sensor nodes and range, which is vital for some plants in order for them to stay alive. 

 
%Detailed hardware setup (processor, RF, sensors, battery, memory, etc.)
\section{Detailed hardware setup}

The wireless network system consists of a Hub and sensor nodes. After careful examination of all available nodes the Multi-standard SensorTag from Texas instruments \cite{TIsensortag} was chosen.


\subsection{Sensor Node - TI Sensor Tag}

The General requirements for the node were design for durability, optimal functionality and user friendly application. This sensor application requires many precise design parameters from the node but those can not all be satisfied. Thus, the general requirements will be discussed and then shown how well the TI sensor tag fulfills them and how it could be added. With the time scope of the project, some of these requirements will not be met, rather a focus will be but on functionality of current components.

Im general the nodes have to be small in size to fit even in small plant pots. They should be battery powered and in order to be able to acquire the necessary data, they need to have sensors in and above the soil. Moreover, the nodes should be waterproof, contain a status-LED and a RF-module for wireless communication with the Hub.

Each sensor node should be able to measure the items listed in Table \ref{list-req} and if the sensor tag fulfills them or not.
\begin{table}[htbp]
	\centering
	\begin{tabular}{lc}
		\toprule
		Requirements & \multicolumn{1}{l}{TI sensor tag} \\ 
		\midrule
		Brightness & yes \\ 
		Temperature & yes \\ 
		Soil humidity & no \\ 
		Air humidity & yes \\ 
		Soil Ph-value & no \\
		\bottomrule 
	\end{tabular}
	\vspace{0.1cm}
	\caption{Requirement compared to the capabilities of the TI sensor tag}
	\label{list-req}
\end{table}


A high sampling rate is not necessary and therefore keeping the sampling rate to a minimum increases battery life substantially. For the same reason it is advisable to decrease the transmission rate by only sending data upon significant change.

\subsubsection{Detailed hardware design}

The sensor tag from TI can measure many things which are useful for the project and was chosen were it also offers various connection methods via BLE, Zigbee, 6LowWPAN and soon wifi. Its battery lasts a year with one second report intervals and is easily replaceable be the user. 

The multi-standard processor, CC2560 \cite{block-cc2650}, connects to ten low power MEMS (Microelectromechanical systems) \cite{MEMS} sensors seen in Figure \ref{fig:sensors}. This makes it possible to be extremely small (micrometers), light weight, mobile and robust.

\begin{figure}[!h]
	\includegraphics[width=\linewidth]{sensors-layout}
	\caption{Block diagram of the MEMSs sensors and the CC2650 processor \cite{block-cc2650}.}
	\label{fig:sensors}
\end{figure}


\subsection{Hub - Raspberry Pi 3}
The Hub has to be able to collect all data provided from the nodes. It can be connected to the buildings power grid and Internet. A user interface (UI) should be digitally accessible via touchscreen, web page and/or app. Additionally the system should be able to inform the user about critical sensor values. To decide whether a value is critical or not, the Hub has to know which type of plant the sensor monitors and then check a Internet database for specific requirements or rely on user defined values.

The Raspberry pi 3, will be referred to as the Pi, is able to connect to the internet via wifi or ethernet without extra adapters. In order for the Pi to communicate with the sensor tag, it has to wirelessly connect to it. There is a BLE module already installed on the Pi board but it doesn't offer the range of the Zigbee \cite{zigbee} which the sensor tag supports. So connecting an extra hardware module which supports Zigbee is required.

\subsubsection{Zigbee - XBee}

A general purpose Zigbee module, Xbee S1 \cite{xbee} seen in Figure \ref{fig:Xbees}, was chosen to communicate with the Pi via serial as they both operate on 3.3V.

\begin{figure}[!h]
	\includegraphics[width=\linewidth]{xbee}
	\caption{Block diagram of the MEMSs sensors and the CC2650 processor \cite{xbee}}
	\label{fig:Xbees}
\end{figure}





\section{Protocol stack}
\subsection{Overview}
The main idea of all system is to create an interface between the wireless sensor network that work via IEEE 802.15.4 and the Internet. This link, made by the raspberry pi, is needed because the data that has to be sense by the wireless sensor network it is needed to be available on a web page for let the user analyze them. 

\subsection{IEEE 802.15.4}
IEEE 802.15.4 is a standard used for Low-power wireless Personal Area Networks (LP-WPANs). It covers physical and media access control layers and there has been released three versions, the first in 2003 and the others in 2006 and 2007 respectively.
\subsubsection{phy characteristic}
In IEEE 802.15.4 the Physical layer operates in 27 channels: 1 in 868MHz, 10 in 902MHz and 16 in 2.4GHz frequency. The data transfer rate up to 250 Kbps.\cite{slide}\\
\subsection{Network Topologies}
In this section, the network architecture of a plant monitoring system will be discussed. \\
Firstly, the IEEE 802.15.4 regulation define two main types of node: the Full Function Device (FFD) and Reduced Function Device (RFD). The RFD differ from the other because is a device with reduced function capabilities, is not able to forward data from one node to another but also cannot be a coordinator for the other node of the network. Therefore, is a device with less resourced that can only send messages to other node. The final network will be composed by a main FFD also called Pan coordinator, other Full Function Devices named co-coordinator and all the other RFD denominated End device. \cite{802-15-4} \\
Secondly, the standard define some topology configuration of a network. The most simple one is the star topology. It is composed by a central pan coordinator that is connected with all other devices. Each device sends data to the pan coordinator and no need to have a node that forwards messages. Moreover, in this type of architecture there are not any node dependence to let data arrive to the main hub.\\
Another type is the tree topology. This has the peculiarity of having a family group of node that send data to a central co-coordinator node which in turn froward the messages to the Pan coordinator. It differ from the previous one by the range of communication that the overall network is able to exploit.
Finally, the peer-to-peer communication is an ad hoc network system where most of the node are FFD. Therefore, each node has not only a central co-coordinator node as it was in tree topology but in here, all the nodes are interconnected with each other creating a network of communication. This system is more reliable from the previous one because, in case of failure of a co-coordinator node, in the previous case we lose all the family of node connected to it. In peer-to-peer, the network is able to re-adapt to this failure using the other communication channel. \cite{slide}\\
The hardware used in the application discussed in this paper is: a Raspberry Pi board with a TelosB board for communication will be the pan coordinator and SensorTag by Texas Instrument for all the other node. The choice of a proper network architecture for a plant monitoring system is particularly related to the place of application of this system. Furthermore, in case of the system is located in a small house then a star topology will be a satisfactory choice. In case of more spacious area where the range between two node is longer, a peer-to-peer topology will be indicated. In case of a multiple leveled building, a pan coordinator on each floor would be required.\\
Nevertheless, because this application uses 6lowPan protocol (will be discussed later on this paper) a mesh topology, previously discussed with peer-to-peer name, will be used.
\\

\subsection{MAC Requirements}
For choosing the best suited MAC-Layer for the system, the following requirements have to be considered:

\begin{itemize}
	\item \textbf{Datarate:}
	Low. All that has to be transmitted by the plants are some sensor values. Therefore a low datarate is sufficient.
	\item \textbf{Network Size:}
	Big. All Nodes have to be connected to the Hub. In order for the system to be able to operate in small apartments as well as in large buildings, it should provide multihop capabilities.
	\item \textbf{Energy Consumption:}
	Very Low. The system consists of several battery powered nodes that can be spread over a large area. Therefore, changing batteries can be very time consuming and should be made as avoidable as possible.
	\item \textbf{Latency:}
	Uncritical. The monitored values are usually changing relatively slow so that real-time measuring is not necessary.
\end{itemize}

While CSMA-Approaches are usually a good choice for low-traffic, multi-hop networks always require some kind of synchronization for making it energy efficient, which is not provided by CSMA.\\
TDMA-Approaches on the other hand are easily applied in multi-hop networks. However, TDMA is relatively inflexible to a changing network topology and generally a bad choice for low-traffic networks.\\
In IEEE 802.15.4 a CSMA/CA is used. The protocol works as follows:
the sender wait until the channel is idle, if it is the case it wait for a random back-off period and then it send the data. The receiver send back an ack if the data where correctly received. Naturally, if the sender do not sense the ack packet, it start to send again the data.\cite{slide}\\





\subsection{6LowPAN}


\subsection{Design of Z-MAC}
\subsubsection{General Idea}
Z-MAC aims to combine advantages of CSMA and TDMA. To achieve this, Z-MAC uses time frames which are divided into slots, just like in TDMA. However, while each node still owns one slot, they are also allowed to transmit data in slots they do not own. To avoid collisions CSMA mechanism are used before transmission with the slight adaption, that is, the contention window is much smaller if a node transmits in its own slot compared to transmission in other slots. Therefore achieving high channel utilization even if some nodes are more active than others, while multi-hop transmission is still possible.\cite{rhee2008z}

\subsubsection{Slot Assignment}
%TODO: second phase name?
Z-MAC consists of two phases: setup and working phase. The setup phase is needed to discover the network topology and assign slots to nodes. Therefore the setup phase should be issued the very first time the system starts and every time there are major changes in the network's topology.\\
It is important for each node to know his one- and two-hop neighbours. Therefore, during setup phase, each node broadcasts a list of its neigbours once per second for thirty seconds. After that, each node chooses one slot as his own using the DRAND (Distributed RAND Algorithm). Once this is done the system is ready to switch to
%TODO: Add second phase name and delete linefeeds
phase.

\subsubsection{Z-MAC Transmission Control}
A node can be in \textit{low contention level} (LCL) or \textit{high contention level} (HCL) mode. By default it is in LCL mode and will only be in HCL if it received an \textit{explicit contention notification} (ECN) message from a one- or two-hop neighbors within the last $t_{ECN}$ period. If a Node is in LCL mode, it is allowed to send at any time slot, following CSMA. In HCL mode only owners of the current slot and their one-hop neighbors are allowed to send at a given time. In order for this to work, each node has to estimate the current contention level of the channel and based on that decide to send ECN messages.

\subsubsection{Z-MAC Reception Control}
Since Z-MAC is based on B-MAC, it also features listening duty cycles separated by check periods. Furthermore, each transmission is preceded by a preamble as long as the check period.

\subsubsection{Synchronization}
Whenever separate nodes want to use different time slots in one frame, they need to synchronize to slot 0. This is achieved by performing global clock synchronization (such as TPSN) in the beginning and later using a low-cost local synchronization protocol. In the case of Z-MAC, synchronization is only needed in HCL mode and only locally, since Z-MAC works just like regular CSMA under low-traffic conditions and HCL is limited to two-hop neighbors.


\section{Network architecture}
Using all the concept described in the previous sections it is possible to describe the plant monitoring system application. All the following steps has been made by using the guide developed by Laurent Deru, Sébastien Dawans and Mathieu Ocana, available in GitHub under permissive 3-clause BSD-style open source license. However, in the tutorial it is used only TelosB node instead of the Texas Instrument Sensor Tags that is used in this development of a plant monitoring system.\\
\begin{figure}[!h]
	\includegraphics[width=\linewidth]{Network}
	\caption{Development of the network}
	\label{fig:Network}
\end{figure}
\subsection{The network}
The network main idea is well depicted in fig\ref{fig:Network}. The raspberry pi act as a six low pan border router (6LBR) and act as a link between the Ethernet and the wireless sensor network. The wireless sensor network is composed by the TelosB that act as a sink for all the other sensor tags. The TelosB sensor is attached via USB to the Raspberry Pi.
On the other hand, the Raspberry Pi is connected to a computer via a normal house router.\cite{6LBR}
The Raspberry Pi connect the two subnets via RPL protocol on the wireless sensor network side and NPD on the Ethernet part.\cite{mode}\\
\subsection{WebServer}




\section{Development}
\subsection{flashing the Texas Instrument Sensor-Tag}
In order to flash the sensor tag, SmartRF Flash Programmer 2 made by Texas Instrument has been used. For flashing the node a sensor tag needs a Debugger DevPack, the debugger allow to communicate to the sensor via USB as it is shown in the picture \ref{fig:debugger}. The 
\begin{figure}[!h]
	\includegraphics[width=\linewidth]{debugger}
	\caption{sensor node connected with the debugger to a computer}
	\label{fig:debugger}
\end{figure} 

\subsection{title}

\section{Hardware and software implementation plan}


\begin{ganttchart}[
	vgrid,
	hgrid
	]{1}{12}
	\gantttitle{Implementation plan - Hardware \& software}{12} \\
	\ganttbar{Task 1}{1}{4} \\
	\ganttbar{Task 2}{5}{6} \\
	\ganttmilestone{M 1}{6} \\
	\ganttbar{Task 3}{7}{11}
	\ganttlink{elem0}{elem1}
	\ganttlink{elem1}{elem2}
	\ganttlink{elem2}{elem3}
\end{ganttchart}
\endgroup



%\begin{figure}[!t]
%\centering
%\includegraphics[width=2.5in]{myfigure}
% where an .eps filename suffix will be assumed under latex, 
% and a .pdf suffix will be assumed for pdflatex; or what has been declared
% via \DeclareGraphicsExtensions.
%\caption{Simulation results for the network.}
%\label{fig_sim}
%\end{figure}

%\begin{figure*}[!t]
%\centering
%\subfloat[Case I]{\includegraphics[width=2.5in]{box}%
%\label{fig_first_case}}
%\hfil
%\subfloat[Case II]{\includegraphics[width=2.5in]{box}%
%\label{fig_second_case}}
%\caption{Simulation results for the network.}
%\label{fig_sim}
%\end{figure*}

%\begin{table}[!t]
%% increase table row spacing, adjust to taste
%\renewcommand{\arraystretch}{1.3}
% if using array.sty, it might be a good idea to tweak the value of
% \extrarowheight as needed to properly center the text within the cells
%\caption{An Example of a Table}
%\label{table_example}
%\centering
%% Some packages, such as MDW tools, offer better commands for making tables
%% than the plain LaTeX2e tabular which is used here.
%\begin{tabular}{|c||c|}
%\hline
%One & Two\\
%\hline
%Three & Four\\
%\hline
%\end{tabular}
%\end{table}
\begin{flushleft}
	
\section{Conclusion}
	The authors hope that this system will help people maintain their apartment green.
	The device described in this paper will be develop in  next two month creating a real prototype.
\end{flushleft}

\printbibliography


% that's all folks
\end{document}


