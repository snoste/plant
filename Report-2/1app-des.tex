\section{Application description}
Keeping plants alive can be a very time consuming task: it requires knowledge about different types of plants as well as periodical checks if these requirements are fulfilled. 
The main idea behind this application is to create a network of nodes, placed directly in the soil of each plant, that, connected to a main hub, helps the user to keep all the plants healthy. Moreover, each nodes has a series of sensor to measure some of the environmental indexes, e.g. temperature or humidity. The main Hub, a device, receives all the sensors data from each node and stores it in a database. This hub uses the contributed data from each plants and checks if it's acceptable for the type of plant the user defined. The costumer can easily view all the sensors information which is stored in the Hub's memory and displayed in a user friendly platform on the web.	\\
This application is suitable for many type of buildings but in this paper it will be assumed to be operating in a large family house with a reasonable garden. The plant monitor system should work indoor and outdoor. Although not all plants get exposed to rain, every node should be waterproof were they are placed directly on the soil of the plant. However, in this project the robust final design will not be implemented in relations to water resistance and more focus is on protocols and application layer.\\
Each node has numerous sensor which will inform the user of necessary actions needed to be taken. In this paper we describe only the sensors available on sensor tag by Texas Instrument. In further development additional sensors could be added to the node.
 \\