\section{Network architecture}
In this section, the network architecture of a plant monitoring system will be discussed. The standard considered now is IEEE 802.15.4.\\
 Firstly, this regulation define two main types of node: the Full Function Device (FFD) and Reduced Function Device (RFD). The RFD differ from the other because is a device with reduced function capabilities, is not able to forward data from one node to another but also cannot be a coordinator for the other node of the network. Therefore, is a device with less resourced that can only send messages to other node. The final network will be composed by a main FFD also called Pan coordinator, other Full Function Devices named co-coordinator and all the other RFD denominated End device. \cite{802-15-4} \\
 Secondly, the standard define some topology configuration of a network. The most simple one is the star topology. It is composed by a central pan coordinator that is connected with all the other devices. Each device send data to the pan coordinator and there is not need to have node that forward messages. Moreover, in this type of architecture there are not any node dependence to let data arrive to the main hub.\\
 Another type is the tree topology. This has the peculiarity of having a family group of node that send data to a central co-coordinator node which in turn froward the messages to the Pan coordinator. It differ from the previous one by the range of communication that the overall network is able to exploit.
 Finally, the peer-to-peer communication is an ad hoc network system where most of the node are FFD. Therefore, each node has not only a central co-coordinator node as it was in tree topology but in here, al the node are much more interconnected with each other creating a network of communications. This system is more reliable from the previous one because, in case of failure of a co-coordinator node, in the previous case we lose all the family of node connected to it. In peer-to-peer, the network is able to re-adapt to this failure using the other communication channel. \cite{slide}\\
 The hardware used in the application discussed in this paper is: a Raspberry Pi board with an XBee board for communication will be the pan coordinator and SensorTag by Texas Instrument for all the other node. The choice of a proper network architecture for a plant monitoring system is particularly related to the place of application of this system. Furthermore, in case the system is located in a small house then a star topology will a satisfactory choice. In case of much more vast spaces where the range between two node is not longer sufficient, a peer-to-peer topology will be indicated. In case of a building distributed in several floor then will be needed to install a pan coordinator to each floor.\\
 Nevertheless, because the communication hardware used in this development is an XBee board, the ZigBee protocol will be used. ZigBee is based on IEEE 802.15.4 and developed by ZigBee alliance. In this protocol we have the same devices as discussed previously: ZigBee coordinator (Pan coordinator), ZigBee Router (co-coordinator) and ZigBee end-devices (RFD devices).\cite{zigbee}\\
 To conclude, the Sensors tag in this application act as ZigBee Router and ZigBee end-device.
\\
