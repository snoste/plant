
%% bare_conf.tex
%% V1.4b
%% 2015/08/26
%% by Michael Shell
%% See:
%% http://www.michaelshell.org/
%% for current contact information.
%%
%% This is a skeleton file demonstrating the use of IEEEtran.cls
%% (requires IEEEtran.cls version 1.8b or later) with an IEEE
%% conference paper.
%%
%% Support sites:
%% http://www.michaelshell.org/tex/ieeetran/
%% http://www.ctan.org/pkg/ieeetran
%% and
%% http://www.ieee.org/

%%*************************************************************************
%% Legal Notice:
%% This code is offered as-is without any warranty either expressed or
%% implied; without even the implied warranty of MERCHANTABILITY or
%% FITNESS FOR A PARTICULAR PURPOSE! 
%% User assumes all risk.
%% In no event shall the IEEE or any contributor to this code be liable for
%% any damages or losses, including, but not limited to, incidental,
%% consequential, or any other damages, resulting from the use or misuse
%% of any information contained here.
%%
%% All comments are the opinions of their respective authors and are not
%% necessarily endorsed by the IEEE.
%%
%% This work is distributed under the LaTeX Project Public License (LPPL)
%% ( http://www.latex-project.org/ ) version 1.3, and may be freely used,
%% distributed and modified. A copy of the LPPL, version 1.3, is included
%% in the base LaTeX documentation of all distributions of LaTeX released
%% 2003/12/01 or later.
%% Retain all contribution notices and credits.
%% ** Modified files should be clearly indicated as such, including  **
%% ** renaming them and changing author support contact information. **
%%*************************************************************************


\documentclass[conference]{IEEEtran}
\usepackage{cite}
\usepackage[pdftex]{graphicx}
 % \graphicspath{{../pdf/}{../jpeg/}}
 % \DeclareGraphicsExtensions{.pdf,.jpeg,.png}
\usepackage{amsmath}
%\usepackage{algorithmic}
%\usepackage{array}
%\usepackage{url}
% correct bad hyphenation here
\hyphenation{op-tical net-works semi-conduc-tor}


\begin{document}
\title{PLANTS\\Wireless Monitoring System}

\author{\IEEEauthorblockN{Filippo Bernardi}
\IEEEauthorblockA{Bologna University, Italy\\
Tongji University, China\\
TU/e, Netherlands\\
Email: filippobernardi@outlook.it}
\and
\IEEEauthorblockN{Snorri Stefánsson}
\IEEEauthorblockA{University of Reykjavik, Iceland\\
TU/e, Netherlands\\
Email: snorriste@gmail.com}
\and
\IEEEauthorblockN{Jonas Wallmeier}
\IEEEauthorblockA{Technical University Ilmenau, Germany\\
TU/e, Netherlands\\
Email: j.a.wallmeier@student.tue.nl}}

% make the title area
\maketitle
Description of a wireless application for monitoring vital signs of plants and for supporting the user to maintain them alive. 
\begin{abstract}

\end{abstract}

% For peerreview papers, this IEEEtran command inserts a page break and
% creates the second title. It will be ignored for other modes.
\IEEEpeerreviewmaketitle



\section{Introduction}
Keeping plants alive can be a very time consuming task: it requires knowledge about different types of plants as well as periodical checks if these requirements are fulfilled. There for it would be much more convenient to have a single device which provides all the necessary information to keep all the plants healthy. This ''Hub'' gets the information wirelessly through a network of sensor nodes which can be placed directly in the soil of each plant.
 
\hfill \today

\section{Design}
\subsection{Sensor Node}
The nodes have to be small in size to fit even in small plant pots. They should be battery powered and in order to be able to acquire the necessary data, they need to have sensors in and above the soil. Moreover, the nodes should be waterproof, contain a status-LED and a RF-module for wireless communication with the Hub.

Each sensor node should be able to measure the following:
\begin{itemize}
	\item Brightness
	\item Temperature
	\item Soil humidity
	\item Air humidity
	\item Ph-Value
\end{itemize}
However, a high sampling rate is not necessary and therefore keeping the sampling rate to a minimum increases battery life substantially. For the same reason it is advisable to decrease the transmission rate by only sending data upon significant change.

\subsection{Hub}
The Hub has to be able to collect all data provided from the nodes. It can be connected to the buildings power grid and internet. A user interface (UI) should be digitally accessible via touchscreen, webpage and/or app. Additionally the system should be able to inform the user about critical sensor values. To decide whether a value is critical or not, the Hub has to either know which type of plant the sensor monitors and then check a internet database for specific requirements or rely on user defined values.
The Hub has to be able to collect all data provided from the nodes. It can be connected to the buildings power grid and Internet. A user interface (UI) should be digitally accessible via touchscreen, web page and/or app. Additionally the system should be able to inform the user about critical sensor values. To decide whether a value is critical or not, the Hub has to either know which type of plant the sensor monitors and then check a Internet database for specific requirements or rely on user defined values.


%\begin{figure}[!t]
%\centering
%\includegraphics[width=2.5in]{myfigure}
% where an .eps filename suffix will be assumed under latex, 
% and a .pdf suffix will be assumed for pdflatex; or what has been declared
% via \DeclareGraphicsExtensions.
%\caption{Simulation results for the network.}
%\label{fig_sim}
%\end{figure}

%\begin{figure*}[!t]
%\centering
%\subfloat[Case I]{\includegraphics[width=2.5in]{box}%
%\label{fig_first_case}}
%\hfil
%\subfloat[Case II]{\includegraphics[width=2.5in]{box}%
%\label{fig_second_case}}
%\caption{Simulation results for the network.}
%\label{fig_sim}
%\end{figure*}

%\begin{table}[!t]
%% increase table row spacing, adjust to taste
%\renewcommand{\arraystretch}{1.3}
% if using array.sty, it might be a good idea to tweak the value of
% \extrarowheight as needed to properly center the text within the cells
%\caption{An Example of a Table}
%\label{table_example}
%\centering
%% Some packages, such as MDW tools, offer better commands for making tables
%% than the plain LaTeX2e tabular which is used here.
%\begin{tabular}{|c||c|}
%\hline
%One & Two\\
%\hline
%Three & Four\\
%\hline
%\end{tabular}
%\end{table}

\begin{flushleft}
	
\section{Conclusion}
The conclusion goes here.

\section*{Acknowledgment}
The authors would like to thank your mum.


	The authors hope that this system will help people maintain their apartment green.
	The device described in this paper will be develop in  next two month creating a real prototype.
\end{flushleft}

\begin{thebibliography}{1}

\bibitem{IEEEhowto:kopka}
H.~Kopka and P.~W. Daly, \emph{A Guide to \LaTeX}, 3rd~ed.\hskip 1em plus
  0.5em minus 0.4em\relax Harlow, England: Addison-Wesley, 1999.

\end{thebibliography}

% that's all folks
\end{document}


