\section{Application description}
Keeping plants alive can be a very time consuming task: it requires knowledge about different types of plants as well as periodical checks if these requirements are fulfilled. 
The main idea behind this application is to create a network of nodes, placed directly in the soil of each plant, that, connected to a main hub, helps the user to keep all the plants healthy. Moreover, each node has a series of sensor to measure some of the environmental indexes, e.g. temperature or humidity. The Main-Hub device receives all the sensors data from each node and stores it in a database. This hub uses the contributed data from each plants and checks if it's acceptable for the type of plant the user defined. Via an user friendly web interface, the costumer can then easily view all information provided by the Hub.\\
The application is suitable for many type of buildings but in this paper it will be assumed to be operating in a large family house with a garden of reasonable size. The plant monitoring system should work indoor and outdoor. Even though plants might not be exposed to rain, all nodes should be waterproof since they are placed directly on the soil of the plant. However, in this project the robust final design will not be implemented in relations to water resistance and more focus is put on protocols and application layer.\\
Each node has numerous sensors which will inform the user of necessary actions needed to be taken. In this paper we describe only the sensors available on Texas Instrument's SensorTag CC2650. In further development additional sensors can we added to the node.
 \\