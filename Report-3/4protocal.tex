\section{Protocol stack}
\subsection{Overview}
The main objective is to create an interface between the wireless sensor network via IEEE 802.15.4 and the internet. This interface, made by the Raspberry pi, is needed because the data that has to be sensed by the wireless sensor network, has to be available on a web page, letting the user analyze them. \\
\subsection{IEEE 802.15.4}
IEEE 802.15.4 is standardized for Low-power wireless Personal Area Networks (LP-WPANs). It covers the physical and medium access control layers and there have been released three versions since 2003.
In IEEE 802.15.4 the Physical layer operates in 27 channels: 1 in 868MHz, 10 in 902MHz and 16 in 2.4GHz frequency. The data transfer rate is up to 250 Kbps.\cite{slide}\\
\subsection{Network Topologies}
In this section, the network architecture of a plant monitoring system will be discussed. \\
Firstly, the IEEE 802.15.4 regulation defines two types of node: the Full Function Device (FFD) and Reduced Function Device (RFD). The RFD differs from the other were it has reduced functional capabilities: it can not forward data from one node to another nor be a coordinator for other nodes of the network. Therefore, it is a device with fewer resources that can only send messages to other nodes. The final network is composed by a main FFD referred to as a Pan coordinator, the Full Function Devices as the co-coordinator and all the other RFD as End device. \cite{802154} \\
Secondly, the standard defines some topology configuration of the network. The most simple one is the star topology. It is composed by a central pan coordinator that is connected with all other devices. Each device sends data to the pan coordinator and no need to have a node that forwards messages. Moreover, in this type of architecture there are not any node dependence to let data arrive to the main hub.\\
Another type is the tree topology. This has the peculiarity of having a family group of node that send data to a central co-coordinator node which in turn froward the messages to the Pan coordinator. The advantage of this topology it the increased range of communication that the overall network is able to exploit.
Finally, the peer-to-peer communication is an ad hoc network where most of the nodes are FFD. Therefore, each node has not only a central co-coordinator but they are also interconnected with each other creating a network of routed communication. This system is more reliable from the previous one were, in the case of a failure of a co-coordinator node, no other nodes are lost compared to tree topology were all child nodes would be lost. In peer-to-peer, the network is able to adapt and fix this failure using the other communication channels.\\
The hardware used in the application discussed in this paper is: a Raspberry Pi board with a TelosB board for communication will be the pan coordinator and SensorTag by Texas Instrument for all the other node. The choice of a proper network architecture for a plant monitoring system is particularly related to the place of application of this system. Furthermore, in case the system is located in a small house the star topology will be a satisfactory choice. In case of more spacious area where the range between two node is longer, a peer-to-peer topology will be preferred. In case of a multiple leveled building, a pan coordinator on each floor would be required.\\
Nevertheless, because this application uses 6lowPan protocol (will be discussed later on this paper) a mesh topology, previously discussed with peer-to-peer name, will be used.
\\



\subsection{MAC Requirements}
For choosing the suitable MAC-Layer for the system, the following requirements have to be considered:

\begin{itemize}
	\item \textbf{Datarate:}
	Low. All that has to be transmitted by the plants are some sensor values. Therefore a low datarate is sufficient.
	\item \textbf{Network Size:}
	Big. All Nodes have to be connected to the Hub. In order for the system to be able to operate in small apartments as well as in large buildings, it should provide multihop capabilities.
	\item \textbf{Energy Consumption:}
	Very Low. The system consists of several battery powered nodes that can be spread over a large area. Therefore, changing batteries can be very time consuming and should be made as avoidable as possible.
	\item \textbf{Latency:}
	Uncritical. The monitored values are usually changing relatively slow so that real-time measuring is not necessary.
\end{itemize}

While CSMA-Approaches are usually a good choice for low-traffic, multi-hop networks always require some kind of synchronization for making it energy efficient, which is not provided by CSMA.\\
TDMA-Approaches on the other hand are easily applied in multi-hop networks. However, TDMA is relatively inflexible to a changing network topology and generally a bad choice for low-traffic networks.\\
In IEEE 802.15.4 CSMA/CA is used. The protocol works as follows:
the sender wait until the channel is idle, if it is the case, the sender wait for a random back-off period and then it send the data. The receiver, if the data are correctly received, send back an acknowledgment packet back to the sender. Naturally, if the sender do not sense the acknowledgment packet, it start to send again the data.\\

\subsection{un-slotted CSMA}
IEEE 802.15.4 can use slotted or un-slotted CSMA procedure. Here the un-slotted procedure will solely be discussed because it is implemented in the plant monitoring system. The un-slotted procedure is as follow: after an initialization phase, the system waits for a random back-off period. Thereafter it uses clear channel assessment (CCA), it checks if the channel is idle or not. If the channel is idle the procedure succeeds otherwise the node backs off. The system repeats from the backoff period until the channel is idle or the backoff procedure has reached the maximum allowed times. This cause a failure in the procedure.\\


\subsection{6LowPAN}
6LowPAN stands for IPv6 over Low Power Wireless Personal Area Networks and it is based on the IEEE 802.15.4 protocol. Ipv6 is a relatively new type of Internet protocol and it has been made for handle the increasing amount of devices that are connected to the Internet. It differs from the IPv4 were it uses a much longer IP \cite{6lowpan}.\\
6LowPAN standards has been developed for connecting embedded devices on the Internet via IPv6.\cite{why}
The main advantage of having an IP in a IEEE 802.15.4 device is that a node is able to use directly an IP network without the use of any other interface. Moreover, a node with an IP address is able to use an pre-built infrastructure. These features meet some general requirements of IoT, Internet of Things: smart devices that collect data and upload them directly to the Internet.
However, there is one main issues in 6loWPAN: the difference packet size between the IEEE 802.15.4 and the IPv6. Firsly, in IEEE 802.15.4 the payload is 102 bytes but actual packet is 127 bytes in total, in these 127 bytes there are 25 bytes of frame overhead that are not part of the data. On the other hand, in IPv6 the packet length is of 1280 bytes. This means that it is necessary to do a fragmentation and a reassembly of the packet for converting the IPv6 packet to an IEEE 802.15.4 format. Moreover, a compression of header field is necessary.\\
To conclude, the 6LowPAN is based on IEEE 802.15.4 standard where a un-slotted CSMA/CA is used. The node architecture used is the mesh type. These two characteristics are a good match, even if there are some constrains between those standards, especially for the plant monitoring system.